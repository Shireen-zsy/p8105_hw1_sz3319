% Options for packages loaded elsewhere
\PassOptionsToPackage{unicode}{hyperref}
\PassOptionsToPackage{hyphens}{url}
%
\documentclass[
]{article}
\usepackage{amsmath,amssymb}
\usepackage{iftex}
\ifPDFTeX
  \usepackage[T1]{fontenc}
  \usepackage[utf8]{inputenc}
  \usepackage{textcomp} % provide euro and other symbols
\else % if luatex or xetex
  \usepackage{unicode-math} % this also loads fontspec
  \defaultfontfeatures{Scale=MatchLowercase}
  \defaultfontfeatures[\rmfamily]{Ligatures=TeX,Scale=1}
\fi
\usepackage{lmodern}
\ifPDFTeX\else
  % xetex/luatex font selection
\fi
% Use upquote if available, for straight quotes in verbatim environments
\IfFileExists{upquote.sty}{\usepackage{upquote}}{}
\IfFileExists{microtype.sty}{% use microtype if available
  \usepackage[]{microtype}
  \UseMicrotypeSet[protrusion]{basicmath} % disable protrusion for tt fonts
}{}
\makeatletter
\@ifundefined{KOMAClassName}{% if non-KOMA class
  \IfFileExists{parskip.sty}{%
    \usepackage{parskip}
  }{% else
    \setlength{\parindent}{0pt}
    \setlength{\parskip}{6pt plus 2pt minus 1pt}}
}{% if KOMA class
  \KOMAoptions{parskip=half}}
\makeatother
\usepackage{xcolor}
\usepackage[margin=1in]{geometry}
\usepackage{color}
\usepackage{fancyvrb}
\newcommand{\VerbBar}{|}
\newcommand{\VERB}{\Verb[commandchars=\\\{\}]}
\DefineVerbatimEnvironment{Highlighting}{Verbatim}{commandchars=\\\{\}}
% Add ',fontsize=\small' for more characters per line
\usepackage{framed}
\definecolor{shadecolor}{RGB}{248,248,248}
\newenvironment{Shaded}{\begin{snugshade}}{\end{snugshade}}
\newcommand{\AlertTok}[1]{\textcolor[rgb]{0.94,0.16,0.16}{#1}}
\newcommand{\AnnotationTok}[1]{\textcolor[rgb]{0.56,0.35,0.01}{\textbf{\textit{#1}}}}
\newcommand{\AttributeTok}[1]{\textcolor[rgb]{0.13,0.29,0.53}{#1}}
\newcommand{\BaseNTok}[1]{\textcolor[rgb]{0.00,0.00,0.81}{#1}}
\newcommand{\BuiltInTok}[1]{#1}
\newcommand{\CharTok}[1]{\textcolor[rgb]{0.31,0.60,0.02}{#1}}
\newcommand{\CommentTok}[1]{\textcolor[rgb]{0.56,0.35,0.01}{\textit{#1}}}
\newcommand{\CommentVarTok}[1]{\textcolor[rgb]{0.56,0.35,0.01}{\textbf{\textit{#1}}}}
\newcommand{\ConstantTok}[1]{\textcolor[rgb]{0.56,0.35,0.01}{#1}}
\newcommand{\ControlFlowTok}[1]{\textcolor[rgb]{0.13,0.29,0.53}{\textbf{#1}}}
\newcommand{\DataTypeTok}[1]{\textcolor[rgb]{0.13,0.29,0.53}{#1}}
\newcommand{\DecValTok}[1]{\textcolor[rgb]{0.00,0.00,0.81}{#1}}
\newcommand{\DocumentationTok}[1]{\textcolor[rgb]{0.56,0.35,0.01}{\textbf{\textit{#1}}}}
\newcommand{\ErrorTok}[1]{\textcolor[rgb]{0.64,0.00,0.00}{\textbf{#1}}}
\newcommand{\ExtensionTok}[1]{#1}
\newcommand{\FloatTok}[1]{\textcolor[rgb]{0.00,0.00,0.81}{#1}}
\newcommand{\FunctionTok}[1]{\textcolor[rgb]{0.13,0.29,0.53}{\textbf{#1}}}
\newcommand{\ImportTok}[1]{#1}
\newcommand{\InformationTok}[1]{\textcolor[rgb]{0.56,0.35,0.01}{\textbf{\textit{#1}}}}
\newcommand{\KeywordTok}[1]{\textcolor[rgb]{0.13,0.29,0.53}{\textbf{#1}}}
\newcommand{\NormalTok}[1]{#1}
\newcommand{\OperatorTok}[1]{\textcolor[rgb]{0.81,0.36,0.00}{\textbf{#1}}}
\newcommand{\OtherTok}[1]{\textcolor[rgb]{0.56,0.35,0.01}{#1}}
\newcommand{\PreprocessorTok}[1]{\textcolor[rgb]{0.56,0.35,0.01}{\textit{#1}}}
\newcommand{\RegionMarkerTok}[1]{#1}
\newcommand{\SpecialCharTok}[1]{\textcolor[rgb]{0.81,0.36,0.00}{\textbf{#1}}}
\newcommand{\SpecialStringTok}[1]{\textcolor[rgb]{0.31,0.60,0.02}{#1}}
\newcommand{\StringTok}[1]{\textcolor[rgb]{0.31,0.60,0.02}{#1}}
\newcommand{\VariableTok}[1]{\textcolor[rgb]{0.00,0.00,0.00}{#1}}
\newcommand{\VerbatimStringTok}[1]{\textcolor[rgb]{0.31,0.60,0.02}{#1}}
\newcommand{\WarningTok}[1]{\textcolor[rgb]{0.56,0.35,0.01}{\textbf{\textit{#1}}}}
\usepackage{graphicx}
\makeatletter
\def\maxwidth{\ifdim\Gin@nat@width>\linewidth\linewidth\else\Gin@nat@width\fi}
\def\maxheight{\ifdim\Gin@nat@height>\textheight\textheight\else\Gin@nat@height\fi}
\makeatother
% Scale images if necessary, so that they will not overflow the page
% margins by default, and it is still possible to overwrite the defaults
% using explicit options in \includegraphics[width, height, ...]{}
\setkeys{Gin}{width=\maxwidth,height=\maxheight,keepaspectratio}
% Set default figure placement to htbp
\makeatletter
\def\fps@figure{htbp}
\makeatother
\setlength{\emergencystretch}{3em} % prevent overfull lines
\providecommand{\tightlist}{%
  \setlength{\itemsep}{0pt}\setlength{\parskip}{0pt}}
\setcounter{secnumdepth}{-\maxdimen} % remove section numbering
\ifLuaTeX
  \usepackage{selnolig}  % disable illegal ligatures
\fi
\usepackage{bookmark}
\IfFileExists{xurl.sty}{\usepackage{xurl}}{} % add URL line breaks if available
\urlstyle{same}
\hypersetup{
  pdftitle={p8105\_hw1\_sz3319},
  pdfauthor={Shiyu Zhang},
  hidelinks,
  pdfcreator={LaTeX via pandoc}}

\title{p8105\_hw1\_sz3319}
\author{Shiyu Zhang}
\date{2024-09-21}

\begin{document}
\maketitle

\subsection{Library Packages for
Coding}\label{library-packages-for-coding}

\begin{Shaded}
\begin{Highlighting}[]
\CommentTok{\# Library necessary packages}
\FunctionTok{library}\NormalTok{(ggplot2)}
\FunctionTok{library}\NormalTok{(tidyverse)}
\end{Highlighting}
\end{Shaded}

\subsection{Problem 1}\label{problem-1}

\subsubsection{a) Load the Dataset}\label{a-load-the-dataset}

\begin{Shaded}
\begin{Highlighting}[]
\CommentTok{\# Load "penguins" dataset}
\FunctionTok{library}\NormalTok{(palmerpenguins)}
\FunctionTok{data}\NormalTok{(}\StringTok{"penguins"}\NormalTok{, }\AttributeTok{package =} \StringTok{"palmerpenguins"}\NormalTok{)}
\end{Highlighting}
\end{Shaded}

\subsubsection{\texorpdfstring{b) Describe \textbf{penguins}
Dataset}{b) Describe penguins Dataset}}\label{b-describe-penguins-dataset}

\begin{Shaded}
\begin{Highlighting}[]
\CommentTok{\# Look at data}
\FunctionTok{tail}\NormalTok{(penguins, }\DecValTok{5}\NormalTok{)}
\NormalTok{skimr}\SpecialCharTok{::}\FunctionTok{skim}\NormalTok{(penguins)}
\end{Highlighting}
\end{Shaded}

The \textbf{penguins} dataset contains data on \textbf{344} observations
and \textbf{8} variables. The mean of flipper length is NA The
\textbf{8} important variables are: \textbf{species, island,
bill\_length\_mm, bill\_depth\_mm, flipper\_length\_mm, body\_mass\_g,
sex, year}. It includes following \textbf{8} important variables:
species, island, bill\_length\_mm, bill\_depth\_mm, flipper\_length\_mm,
body\_mass\_g, sex, year.

\subsubsection{c) Creat and Save a
Scatterplot}\label{c-creat-and-save-a-scatterplot}

\begin{Shaded}
\begin{Highlighting}[]
\FunctionTok{library}\NormalTok{(ggplot2)}
\FunctionTok{ggplot}\NormalTok{(}\AttributeTok{data =}\NormalTok{ penguins, }\FunctionTok{aes}\NormalTok{(}\AttributeTok{x =}\NormalTok{ bill\_length\_mm, }\AttributeTok{y =}\NormalTok{ flipper\_length\_mm, }\AttributeTok{color =}\NormalTok{ species)) }\SpecialCharTok{+}
  \FunctionTok{geom\_point}\NormalTok{(}\AttributeTok{na.rm =} \ConstantTok{TRUE}\NormalTok{) }\SpecialCharTok{+}
  \FunctionTok{labs}\NormalTok{(}\AttributeTok{title =} \StringTok{"Flipper Length vs Bill Length in Different Species of Penguins"}\NormalTok{,}
       \AttributeTok{x =} \StringTok{"Bill Length (mm)"}\NormalTok{,}
       \AttributeTok{y =} \StringTok{"Flipper Length (mm)"}\NormalTok{) }\SpecialCharTok{+}
  \FunctionTok{theme\_minimal}\NormalTok{()}
\end{Highlighting}
\end{Shaded}

\includegraphics{p8105_hw1_sz3319_files/figure-latex/scatterplot-1.pdf}

\begin{Shaded}
\begin{Highlighting}[]
\FunctionTok{ggsave}\NormalTok{(}\StringTok{"penguins\_scatter\_plot.pdf"}\NormalTok{, }\AttributeTok{height =} \DecValTok{4}\NormalTok{, }\AttributeTok{width =} \DecValTok{6}\NormalTok{)}
\end{Highlighting}
\end{Shaded}

\subsection{Problem 2}\label{problem-2}

\subsubsection{a) Create the Data}\label{a-create-the-data}

\begin{Shaded}
\begin{Highlighting}[]
\CommentTok{\# For reproducibility}
\FunctionTok{set.seed}\NormalTok{(}\DecValTok{1234}\NormalTok{)}

\CommentTok{\# a random sample of size 10 from a standard Normal distribution}
\NormalTok{normal\_sample }\OtherTok{\textless{}{-}} \FunctionTok{rnorm}\NormalTok{(}\DecValTok{10}\NormalTok{)}

\CommentTok{\# a logical vector indicating whether elements of the sample are greater than 0}
\NormalTok{logical\_vector }\OtherTok{\textless{}{-}}\NormalTok{ normal\_sample }\SpecialCharTok{\textgreater{}} \DecValTok{0}

\CommentTok{\# a character vector of length 10}
\NormalTok{character\_vector }\OtherTok{\textless{}{-}} \FunctionTok{sample}\NormalTok{(letters, }\DecValTok{10}\NormalTok{, }\AttributeTok{replace =} \ConstantTok{TRUE}\NormalTok{)}

\CommentTok{\# a factor vector of length 10, with 3 different factor “levels”}
\NormalTok{factor\_vector }\OtherTok{\textless{}{-}} \FunctionTok{factor}\NormalTok{(}\FunctionTok{sample}\NormalTok{(}\FunctionTok{c}\NormalTok{(}\StringTok{"A"}\NormalTok{, }\StringTok{"B"}\NormalTok{, }\StringTok{"C"}\NormalTok{), }\DecValTok{10}\NormalTok{, }\AttributeTok{replace =} \ConstantTok{TRUE}\NormalTok{))}
\end{Highlighting}
\end{Shaded}

\subsubsection{b) Create a Data Frame}\label{b-create-a-data-frame}

\begin{Shaded}
\begin{Highlighting}[]
\CommentTok{\# create data frame "df"}
\NormalTok{df }\OtherTok{\textless{}{-}} \FunctionTok{tibble}\NormalTok{(}
  \AttributeTok{normal =}\NormalTok{ normal\_sample,}
  \AttributeTok{logical =}\NormalTok{ logical\_vector,}
  \AttributeTok{character =}\NormalTok{ character\_vector,}
  \AttributeTok{factor =}\NormalTok{ factor\_vector}
\NormalTok{)}
\end{Highlighting}
\end{Shaded}

\subsubsection{c) Calculate the mean of different
variables}\label{c-calculate-the-mean-of-different-variables}

\begin{Shaded}
\begin{Highlighting}[]
\CommentTok{\# calculate the mean of normal sample}
\NormalTok{mean\_normal }\OtherTok{\textless{}{-}} \FunctionTok{mean}\NormalTok{(}\FunctionTok{pull}\NormalTok{(df, normal))}
\NormalTok{mean\_normal}
\end{Highlighting}
\end{Shaded}

\begin{verbatim}
## [1] -0.3831574
\end{verbatim}

\begin{Shaded}
\begin{Highlighting}[]
\CommentTok{\#calculate the mean of }
\NormalTok{mean\_logical }\OtherTok{\textless{}{-}} \FunctionTok{mean}\NormalTok{(}\FunctionTok{pull}\NormalTok{(df, logical))}
\NormalTok{mean\_logical}
\end{Highlighting}
\end{Shaded}

\begin{verbatim}
## [1] 0.4
\end{verbatim}

\begin{Shaded}
\begin{Highlighting}[]
\CommentTok{\#calculate the mean of }
\NormalTok{mean\_character }\OtherTok{\textless{}{-}} \FunctionTok{mean}\NormalTok{(}\FunctionTok{pull}\NormalTok{(df, character))}
\NormalTok{mean\_character}
\end{Highlighting}
\end{Shaded}

\begin{verbatim}
## [1] NA
\end{verbatim}

\begin{Shaded}
\begin{Highlighting}[]
\CommentTok{\#calculate the mean of }
\NormalTok{mean\_factor }\OtherTok{\textless{}{-}} \FunctionTok{mean}\NormalTok{(}\FunctionTok{pull}\NormalTok{(df, factor))}
\NormalTok{mean\_factor}
\end{Highlighting}
\end{Shaded}

\begin{verbatim}
## [1] NA
\end{verbatim}

The result shows that:

\begin{itemize}
\tightlist
\item
  The mean of the \textbf{normal sample} is -0.3831574.
\item
  The mean of \textbf{logical vectors} is 0.4, for the \texttt{TRUE} is
  \texttt{1} and \texttt{FALSE} is \texttt{0}.
\item
  The mean of characters and factors can't be calculated for they are
  not numeric data.
\end{itemize}

\subsubsection{d) Convert Variables}\label{d-convert-variables}

\begin{Shaded}
\begin{Highlighting}[]
\CommentTok{\# Convert logical to numeric}
\NormalTok{logi\_num }\OtherTok{\textless{}{-}} \FunctionTok{as.numeric}\NormalTok{(}\FunctionTok{pull}\NormalTok{(df, logical))}
\NormalTok{logi\_num}
\end{Highlighting}
\end{Shaded}

\begin{verbatim}
##  [1] 0 1 1 0 1 1 0 0 0 0
\end{verbatim}

\begin{Shaded}
\begin{Highlighting}[]
\CommentTok{\# Convert character to numeric}
\NormalTok{cha\_num }\OtherTok{\textless{}{-}} \FunctionTok{as.numeric}\NormalTok{(}\FunctionTok{pull}\NormalTok{(df, character))}
\NormalTok{cha\_num}
\end{Highlighting}
\end{Shaded}

\begin{verbatim}
##  [1] NA NA NA NA NA NA NA NA NA NA
\end{verbatim}

\begin{Shaded}
\begin{Highlighting}[]
\CommentTok{\# Convert factor to numeric}
\NormalTok{fac\_num }\OtherTok{\textless{}{-}} \FunctionTok{as.numeric}\NormalTok{(}\FunctionTok{pull}\NormalTok{(df, factor))}
\NormalTok{fac\_num}
\end{Highlighting}
\end{Shaded}

\begin{verbatim}
##  [1] 3 3 1 2 1 2 2 3 3 3
\end{verbatim}

The result shows that:

\begin{itemize}
\tightlist
\item
  The character variable \textbf{cannot} be changed into numeric
  variable, and the output are \texttt{NA}s.
\item
  The factor variable can be changed in to numeric variable. The
  \texttt{A}, \texttt{B}, \texttt{C} are converted into \texttt{1},
  \texttt{2}, \texttt{3}. However, they only randomly convert into
  numbers, so they may not be the real \texttt{Level}.
\end{itemize}

\subsubsection{The result explains
that:}\label{the-result-explains-that}

\begin{itemize}
\tightlist
\item
  The reason of the failure of calculating the mean for character and
  factor.
\item
  The mean of logical vectors can be calculated. When use
  \texttt{as.numeric()}, \texttt{TRUE} will be converted to 1,
  \texttt{FALSE} will be converted to 0. As a result, the mean can be
  calculated.
\item
  For the character variable, when using \texttt{as.numeric()}, all
  elements will return as \texttt{NA}, so there is no meaningful numeric
  values.
\item
  The factor variable can converted to numeric variable, but the numbers
  are only \textbf{code}s, not numbers.
\end{itemize}

\end{document}
